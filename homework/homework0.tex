\documentclass[letter]{article}
\usepackage{amsmath}
\usepackage{amsfonts}
\usepackage{amssymb}
\usepackage{ifthen}
\usepackage{fancyhdr}
\usepackage[hidelinks]{hyperref}
\usepackage{xcolor}
\hypersetup{
    colorlinks,
    linkcolor={red!50!black},
    citecolor={blue!50!black},
    urlcolor={blue!80!black}
}

\definecolor{deepblue}{rgb}{0,0,0.5}
\definecolor{deepred}{rgb}{0.6,0,0}
\definecolor{deepgreen}{rgb}{0,0.5,0}

\usepackage{listings}

% Default fixed font does not support bold face
\DeclareFixedFont{\ttb}{T1}{txtt}{bx}{n}{12} % for bold
\DeclareFixedFont{\ttm}{T1}{txtt}{m}{n}{12}  % for normal

% Python style for highlighting
\newcommand\pythonstyle{\lstset{
language=Python,
basicstyle=\ttm,
otherkeywords={self},             % Add keywords here
keywordstyle=\ttb\color{deepblue},
emph={MyClass,__init__},          % Custom highlighting
emphstyle=\ttb\color{deepred},    % Custom highlighting style
stringstyle=\color{deepgreen},
frame=tb,                         % Any extra options here
showstringspaces=false            % 
}}


% Python environment
\lstnewenvironment{python}[1][]
{
\pythonstyle
\lstset{#1}
}
{}

% Python for external files
\newcommand\pythonexternal[2][]{{
\pythonstyle
\lstinputlisting[#1]{#2}}}

% Python for inline
\newcommand\pythoninline[1]{{\pythonstyle\lstinline!#1!}}


%%%
% Set up the margins to use a fairly large area of the page
%%%
\oddsidemargin=.2in
\evensidemargin=.2in
\textwidth=6in
\topmargin=0in
\textheight=9.0in
\parskip=.07in
\parindent=0in
\pagestyle{fancy}

%%%
% Set up the header
%%%
\newcommand{\setheader}[6]{
	\lhead{{\sc #1}\\{\sc #2} ({\small \it \today})}
	\rhead{
		{\bf #3} 
		\ifthenelse{\equal{#4}{}}{}{(#4)}\\
		{\bf #5} 
		\ifthenelse{\equal{#6}{}}{}{(#6)}%
	}
}

\makeatletter
\newcommand{\escapeus}{\begingroup\@makeother\_\@escapeus}
\newcommand*{\@escapeus}[1]{#1\endgroup}
\makeatother

%%%
% Set up some shortcut commands
%%%
\newcommand{\R}{\mathbb{R}}
\newcommand{\N}{\mathbb{N}}
\newcommand{\Z}{\mathbb{Z}}
\newcommand{\Proj}{\mathrm{proj}}
\newcommand{\Perp}{\mathrm{perp}}
\newcommand{\proj}{\mathrm{proj}}
\newcommand{\Span}{\mathrm{span}}
\newcommand{\Null}{\mathrm{null}}
\newcommand{\Rank}{\mathrm{rank}}
\newcommand{\mat}[1]{\begin{bmatrix}#1\end{bmatrix}}
\newcommand{\var}[1]{{$\langle$\it #1$\rangle$}}
\newcommand{\Code}[1]{\texttt{\escapeus #1}}

%%%
% This is where the body of the document goes
%%%
\begin{document}
	\setheader{MAT335}{Homework 0}{Due: 11:59pm January 12}{}{}{}
	In this class, we'll be doing a lot of math, but we'll also be 
	focusing on our writing and communication skills as well as using computers
	to simulate results and make pictures.  This homework is to help familiarize 
	you with the tools we'll be using. In particular, your math writeups will
	be completed in \LaTeX and your simulations will be done in the {\tt Python}
	programming language.

	In this course, you'll be doing \emph{ a lot} of googling, trying to get \LaTeX{} and
	{\tt Python} to do what you want. Don't worry about making mistakes or trying something
	that you don't understand. Dive right in!


	You should submit your answers to parts 1 and 2 to Crowdmark.

	\begin{enumerate}
		\item Prepare a 1-page \LaTeX{}\footnote{ \LaTeX{} is a programming language designed for typsetting
			mathematics. It is the language used to make many of your textbooks, and is the standard for
			Math/CS writeups (but there are also libraries for econ, chemistry, and other disciplines!).} document that has
		\begin{enumerate}
			\item A one-to-two paragraph math biography. Explain how you feel about math: Have you always felt that way?
				Have you had experiences that brought you to like math more? Less?
			\item A footnote (you can use the \verb|\footnote{}| command).
			\item A figure (if you're feeling adventurous, you can use {\tt tikz} to draw your figure; otherwise,
				you can just include an image).
			\item Your two favorite equations, one in \emph{inline} and \emph{display} style. (Inline is when
				they show up in a paragraph by using \verb|$...$|, display is when the math forms
				its own paragraph by using \verb|\[...\]|.)
			\item A $2\times 3$ matrix.
		\end{enumerate}

		You can either install \LaTeX{} or use a free, online version available at \url{https://overleaf.com}. Please
		see \url{https://www.overleaf.com/learn/latex/Creating_a_document_in_LaTeX} for a basic introduction to \LaTeX{}.
		For a more advanced introduction, you can read \url{https://tobi.oetiker.ch/lshort/lshort.pdf}.

		You can find a \LaTeX{} template for writing your homeworks here: 
		\url{https://github.com/siefkenj/2020-MAT-335-webpage/raw/master/homework/sample-homework.tex}



		\item Do the following programming exercises in a \url{https://utoronto.syzygy.ca} Jupyter notebook\footnote{ Jupyter
			notebooks provide an interactive programming environment where you can write {\tt Python} code and get 
			the results immediately.}.

			To ``Do'' a programming exercise means (i) to write the required code \emph{with comments} explaining
			any complicated parts, (ii) execute the code on appropriate examples and show their outputs, and 
			(iii) upload the result to Crowdmark. You can do a {\it File>Print Preview} in your Jupyter notebook
			and then {\it Print to file} to save your notebook output as a PDF. That PDF can be uploaded to Crowdmark.
		\begin{enumerate}
			\item Clone the {\tt homework0-intro.ipynb} into your syzygy account by clicking the link:
				\url{https://utoronto.syzygy.ca/jupyter/user-redirect/git-pull?repo=https://github.com/siefkenj/2020-MAT-335-webpage&subPath=homework/homework0-intro.ipynb}

				Alternatively, if you have installed {\tt Python} and Jupyter notebooks on your local computer, you may
				clone the notebook from \url{https://github.com/siefkenj/2020-MAT-335-webpage/tree/master/homework}
			\item Write a function called {\tt hi()} which takes a first name and a last name and prints 
				\begin{quote}{\tt Hi \var{first name}. What is the origin of '\var{last name}'?}\end{quote}
			\item Write a function \verb|square_the_list()| which inputs a list of numbers and returns
				a list of each of the numbers squared.
			\item Use {\tt numpy} and {\tt matplotlib} to produce a graph of the function $f(x)=x^2$ and the tangent
				line to $f$ at $x=1$ on the same grid. Color the graph of $f$ red (any shade) and the graph of the
				tangent line blue (again, any shade you like!). 
			\item Make a function \verb|is_prime()| that returns $1$ if the input is prime and $0$ otherwise.
			\item Use {\tt \var{axis}.imshow} to make a $50\times 50$ black-and-white image. The $n$th square in your
				image should be black if the number $n$ is not prime and white if the number $n$ is prime. Index your image
				in the following way: the square at coordinates $(i,j)$ corresponds to the number $i+n*j$. 
				And remember, in {\tt Python} the first element of a list is $0$ (not $1$!).
		\end{enumerate}
	\end{enumerate}

\end{document}
