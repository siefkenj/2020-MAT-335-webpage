\documentclass[letter]{article}
\usepackage{amsmath}
\usepackage{amsfonts}
\usepackage{amssymb}
\usepackage{bm}
\usepackage{ifthen}
\usepackage{fancyhdr}
\usepackage{graphicx}
\usepackage[hidelinks]{hyperref}
\usepackage{xcolor}
\hypersetup{
    colorlinks,
    linkcolor={red!50!black},
    citecolor={blue!50!black},
    urlcolor={blue!80!black}
}

\definecolor{deepblue}{rgb}{0,0,0.5}
\definecolor{deepred}{rgb}{0.6,0,0}
\definecolor{deepgreen}{rgb}{0,0.5,0}

\usepackage{listings}

% Default fixed font does not support bold face
\DeclareFixedFont{\ttb}{T1}{txtt}{bx}{n}{12} % for bold
\DeclareFixedFont{\ttm}{T1}{txtt}{m}{n}{12}  % for normal

% Python style for highlighting
\newcommand\pythonstyle{\lstset{
language=Python,
basicstyle=\ttm,
otherkeywords={self},             % Add keywords here
keywordstyle=\ttb\color{deepblue},
emph={MyClass,__init__},          % Custom highlighting
emphstyle=\ttb\color{deepred},    % Custom highlighting style
stringstyle=\color{deepgreen},
frame=tb,                         % Any extra options here
showstringspaces=false            % 
}}


% Python environment
\lstnewenvironment{python}[1][]
{
\pythonstyle
\lstset{#1}
}
{}

% Python for external files
\newcommand\pythonexternal[2][]{{
\pythonstyle
\lstinputlisting[#1]{#2}}}

% Python for inline
\newcommand\pythoninline[1]{{\pythonstyle\lstinline!#1!}}


%%%
% Set up the margins to use a fairly large area of the page
%%%
\oddsidemargin=.2in
\evensidemargin=.2in
\textwidth=6in
\topmargin=-.5in
\textheight=9in
\parskip=.07in
\parindent=0in
\pagestyle{fancy}

%%%
% Set up the header
%%%
\newcommand{\setheader}[6]{
	\lhead{{\sc #1}\\{\sc #2} ({\small \it \today})}
	\rhead{
		{\bf #3} 
		\ifthenelse{\equal{#4}{}}{}{(#4)}\\
		{\bf #5} 
		\ifthenelse{\equal{#6}{}}{}{(#6)}%
	}
}

\makeatletter
\newcommand{\escapeus}{\begingroup\@makeother\_\@escapeus}
\newcommand*{\@escapeus}[1]{#1\endgroup}
\makeatother

%%%
% Set up some shortcut commands
%%%
\newcommand{\R}{\mathbb{R}}
\newcommand{\N}{\mathbb{N}}
\newcommand{\Z}{\mathbb{Z}}
\newcommand{\Proj}{\mathrm{proj}}
\newcommand{\Perp}{\mathrm{perp}}
\newcommand{\proj}{\mathrm{proj}}
\newcommand{\Span}{\mathrm{span}}
\newcommand{\Null}{\mathrm{null}}
\newcommand{\Rank}{\mathrm{rank}}
\newcommand{\mat}[1]{\begin{bmatrix}#1\end{bmatrix}}
\newcommand{\var}[1]{{$\langle$\it #1$\rangle$}}
\newcommand{\Code}[1]{\texttt{\escapeus #1}}
\DeclareMathOperator{\Tr}{Tr}

%%%
% This is where the body of the document goes
%%%
\begin{document}
	\setheader{MAT335}{Homework 3}{Due: 11:59pm March 15}{}{}{}

	\begin{enumerate}
		\item Consider the vector field $\vec F(x,y) = (x+y^2,-y)$ and its associated continuous dynamical system $(W^t, \R^2)$.
			\begin{enumerate}
				\item For each of the following functions, show whether or
					not it is a flow for $(W^t,\R^2)$.
					\begin{enumerate}
						\item $\vec \varphi(t) = (-e^{-2t},\sqrt{3}e^{-t})$
						\item $\vec \varphi(t) = (-e^{-4t},\sqrt{3}e^{-2t})$
						\item $\vec \varphi(t) = (\frac{4}{3}e^t-\frac{1}{3}e^{-2t},e^{-t})$
						\item $\vec \varphi(t) = (e^t-e^{-2t},e^{-t})$
					\end{enumerate}
				\item A \emph{constant flow} is a flow $\vec\varphi$ such that $\vec\varphi(t_1)=\vec \varphi(t_2)$ for
					all $t_1,t_2\in \R$.  Find all constant flows for $\vec F$. (Hint: think about $\vec\varphi\,'$ in
					this situation.)
				\item Classify all fixed points of $(W^t,\R^2)$ as stable or unstable.
				\item Find \emph{all} flows of $(W^t,\R^2)$ that pass through the point $(1,0)$.  Remember,
					if $\vec\varphi$ is a flow with this property, it is not a requirement that $\vec\varphi(0)=(1,0)$,
					only that $\vec \varphi(t_0)=(1,0)$ for some $t_0$.
				\item We call the flows $\vec\varphi_1$
					and $\vec \varphi_2$ time shifts of each other if $\vec \varphi_1(t)=\vec \varphi_2(t+t_0)$ for
					some $t_0$. Show that the flows from part (d) are time shifts of each other.  
			\end{enumerate}


		\item Let's explore the question of whether every discrete dynamical system can be
			described as the time-$1$ map of a continuous dynamical system.

			Let $(W^t,X)$ be a continuous dynamical system, and let $(T,X)$ be its time-$1$ map.
			That is, $T(x) = W^1(x)$.

			The point $x\in X$ is called \emph{periodic} for $(W^t,X)$ if there exists a $t\neq 0$ so that $W^t(x)=x$ and
			is called \emph{periodic} for $(T,X)$ if there exists an $i\neq 0$ so that $T^ix=x$. In both cases, the minimum
			$t$ or $i$ such $W^t(x)=x$ or $T^i(x)=x$ is called the \emph{period} of $x$.
			\begin{enumerate}
				\item Suppose $x\in X$ is a \emph{periodic point} for $(T,X)$. Must $x$ be a periodic point for $(W^t,X)$?
				\item Suppose $x\in X$ is a \emph{periodic point} for $(W^t,X)$. Must $x$ be a periodic point for $(T,X)$?
				\item Show that if $x$ is a point of period at least $2$ for $(T,X)$, then there are infinitely many periodic
					points for $(T,X)$ \emph{of the same period}.
				\item Consider the \emph{logistic map} $T:[0,1]\to[0,1]$ defined by $x\mapsto rx(1-x)$. Show that
					when $r=\frac{19}{6}$ the logistic map has exactly two points of period 2. (Hint: use a computer
					algebra system to solve any nasty equations you come across!)
				\item Are all discrete dynamical systems time-$1$ maps to continuous dynamical systems? Justify your answer.
			\end{enumerate}

		\item The function $G:\R^n\to\R^m$ is called \emph{affine} if there exists a matrix $A$ and
			a vector $\vec p$ so that $G(\vec x) = A\vec x+\vec p$ for all $\vec x$.

			Let $F:\R^n\to\R^m$. A \emph{first-order approximation} to $F$ at the point $\vec w\in \R^n$ is an
			affine function $L:\R^n\to\R^m$ satisfying
			\[
				\lim_{\|\vec x\|\to 0} \frac{F(\vec w+\vec x)-L(\vec w+\vec x)}{\|\vec x\|} = 0.
			\]
			\begin{enumerate}
				\item Let $F:\R\to\R$ be defined by $x\mapsto x^2$. Find a first-order approximation, $L_0$, 
					to $F$ at $0$, and a first-order approximation, $L_2$, to $F$ at $2$.
				\item Let $F:\R^2\to\R^2$ be defined by $\mat{x\\y}\mapsto \mat{a&b\\c&d}\mat{x\\y}$. Define
					$L_{\vec w}$ to be the first-order approximation of $F$ at the point $\vec w$. Find $L_{\vec w}$.
				\item Let $L$ be an affine function and let $L_{\vec w}$ be the
					first-order approximation to $L$ at $\vec w$.
					Prove that $L=L_{\vec w}$ regardless of $\vec w$.
				\item Prove that if $(W^t,\R^n)$ is a continuous dynamical system with velocities given by
					$V(\vec x)=A\vec x$ for some matrix $A$, then the point $\vec x\in \R^n$ is stable under $W^t$
					if and only if the point $\vec 0$ is stable under $W^t$.

				\item From calculus, you know that if $f:\R\to\R$ is differentiable, then $L(w+x)=f'(w)(x)+f(w)$ is a
					first-order approximation to $f$ at $w$, where $f'$ is the derivative of $f$. Use this knowledge
					to find a first-order approximation to $F(x,y) = (x+y^2,-y)$ at the points $(1,1)$ and $(1,0)$.
				\item Let $(W^t,\R^2)$ be the continuous dynamical system that flows along $F(x,y)=(x+y^2,-y)$.
					Classify $(1,1)$ and $(1,0)$ as stable or unstable. Justify your answer.
			\end{enumerate}
		\item Stability/instability describes how points behave under a dynamical system. Let's take a moment to think about
			how \emph{volumes/areas} behave.

			For this problem, you may use any facts you know about the determinant without justification (so long as they're true\ldots).

			\begin{enumerate}
				\item The \emph{trace} of a square matrix $X$, denoted $\Tr(X)$,
					is the sum of its diagonal matrices. Let $A=[\vec a_1|\cdots|\vec a_n]$
					be a matrix with columns $\vec a_1,\ldots,\vec a_n$.  Let $E_i$ be the identity matrix with
					the $i$th column replaced with $\vec a_i$. Show that
					\[
						\Tr(A) = \sum \det(E_i).
					\]
				\item Let $(W^t,\R^2)$ be the continuous dynamical system which flows vectors along the vector field given
					by $A=\mat{a&b\\c&d}$.

					Write out the limit definition of the derivative $\frac{\partial W^t}{\partial t}$ at time $t=0$.
					How does this derivative relate to $A$?
				\item Write down a first-order approximation to $W^t$ \emph{with respect to time} at $t=0$.

					\emph{Hint: the hardest part of this question is figuring out what the previous sentence actually
					means. Don't get discouraged!}


			\end{enumerate}


	\end{enumerate}


	\subsection*{Programming Problems}
	For the programming problems, please use the Jupyter notebook available at

	\url{https://utoronto.syzygy.ca/jupyter/user-redirect/git-pull?repo=https://github.com/siefkenj/2020-MAT-335-webpage&subPath=homework/homework2-exercises.ipynb}

	Make sure to comment your code and use ``Markdown'' style cells to explain your answers.

	\begin{enumerate}

		\item xxx
	\end{enumerate}



\end{document}
