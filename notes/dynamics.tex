
% WARNING!  Do not type any of the following 10 characters except as directed:
%                &   $   #   %   _   {   }   ^   ~   \   
%
%%
%%  default option for pdfx.sty  if not specified on the command-line.
\providecommand{\pdfxopt}{a-1b}
%% 
%%  Use  {filecontents}  for the  .xmpdata file before input encoding is specified.
%%
%%%%%%%%%%%%%%%%%%%%%%%%%%%%%%%%%%%%%%%%%%%%%%%%%%%
\begin{filecontents*}{\jobname.xmpdata}
	% a macro definition, used below
	\pdfxEnableCommands{% simple macro definitions can be provided everything expands to characters
	 \def\RossPete{Ross \& Pete}
	 }
	\Title{Linear Algebra (\jobname)}%  *not* set by LaTeX's  \title
	\Author{Jason Siefken\sep et al.}% *not* set by LaTeX's \author
	\Subject{Linear Algebra textbook/workbook}
	\Keywords{linear algebra\sep vectors\sep mathematics\sep textbook} 
	\Org{University of Toronto}
	\CreatorTool{LaTeX + pdfx.sty with options \pdfxopt}
	\Copyright{Jason Siefken}
	\WebStatement{https://github.com/siefkenj/IBLLinearAlgebra/}% should be URL to copyright statement on the web
	\CoverDisplayDate{2019}
	\CoverDate{2019-08-014}%  must be in format  YYYY-MM-DD  or  YYYY-MM
	\Doi{0.0.0.0}%
	%
	% setting the color profile, these reproduce the defaults; use your own, if required 
	%
	% RGB is used with PDF/A (4 parameters):
	\setRGBcolorprofile{sRGB_IEC61966-2-1_black_scaled.icc}{sRGB_IEC61966-2-1_black_scaled}{sRGB IEC61966 v2.1 with black scaling}{http://www.color.org}
	%
	%  For Adobe Color Profiles, set the directory for your system
	%
	%  e.g.  on Mac OS X
	%  What is it under Windows ?
	%
	\gdef\ColorProfileDir{/Library/Application Support/Adobe/Color/Profiles/Recommended/}
	% 
	%  For available profiles, see file  AdobeColorProfiles.tex
	%  For PDF/X-4p or PDF/X-5pg   see file  AdobeExternalProfiles.tex  
	%
	%  Now you can use the macros defined in those files:
	 \FOGRAXXXIX 
	%
	% or CMYK is used with  PDF/X (4 parameters)
	% \setCMYKcolorprofile{\ColorProfileDir coated_FOGRA39L_argl.icc}{Coated FOGRA39}{FOGRA39 (ISO Coated v2 300\%\space (ECI))}{http://www.color.org}
\end{filecontents*}

\documentclass{problemset}


% pdfx will set color profile etc. information appropriately, so the pdf renders
% consistently across devices. But, it doesn't work with the xelatex-based tectonics
\usepackage{ifxetex}
	\usepackage[utf8]{inputenc}
\ifxetex
\else
	\usepackage[a-3u]{pdfx}
	\hypersetup{hidelinks=true, linkcolor = {0 0 1} }
\fi

%%%
% import all needed packages and macros
%%%
\usepackage[yyyymmdd]{datetime}
\input{preamble.tex}

% in non-xelatex engines, hyperref is loaded by `pdfx`. If `pdfx` is not loaded, load it here.
\ifxetex
	\usepackage{hyperref}
	\hypersetup{hidelinks=true, linkcolor = {0 0 1} }
\else
\fi

%%%
% Set up the footers to have the correct copyright notices
%%%

\fancypagestyle{siefken}{%
	\rfoot{\footnotesize\it \copyright\,Jason Siefken, 2015--2019 \ \makebox(30,5){\includegraphics[height=1.2em]{by-sa.pdf}}}
	\lfoot{}
	\renewcommand{\headrulewidth}{0pt}
}
\fancypagestyle{iola}{%
	\rfoot{\footnotesize\it \copyright\,IOLA Team \url{iola.math.vt.edu} \ \makebox(30,5){\includegraphics[height=2.2em]{images/iolalogo.png}}}
	\lfoot{}
	\renewcommand{\headrulewidth}{0pt}
}

\DeclareDocumentEnvironment{iola}{o}{%
	\newpage
	\pagestyle{iola}
}{%
	\newpage
}


%%
% Allow hiding of environments
%%
\usepackage{environ}% http://ctan.org/pkg/environ
\makeatletter
\newcommand{\voidenvironment}[1]{%
  \expandafter\providecommand\csname env@#1@save@env\endcsname{}%
  \expandafter\providecommand\csname env@#1@process\endcsname{}%
  \@ifundefined{#1}{}{\RenewEnviron{#1}{}}%
}
\makeatother
% allow pagebreaks that only display in `standard` mode
\newcommand{\displayonlynewpage}{\begin{displayonly}\newpage\end{displayonly}}
% allow pagebreaks that only display in `book` mode
\newcommand{\bookonlynewpage}{\begin{bookonly}\newpage\end{bookonly}}

%
% Set up the three render modes: standard, instructor, and solutions.
% These render with varying amounts of extra data (like solutions and notes)
%
\newtoggle{instructor}
\newtoggle{standard}
\newtoggle{solutions}
\newtoggle{book}
\newcommand{\setinstructor}{
	\toggletrue{instructor}
	\togglefalse{standard}
	\togglefalse{solutions}
	\togglefalse{book}
}
\newcommand{\setstandard}{
	\togglefalse{instructor}
	\toggletrue{standard}
	\togglefalse{solutions}
	\togglefalse{book}
}
\newcommand{\setsolutions}{
	\togglefalse{instructor}
	\togglefalse{standard}
	\toggletrue{solutions}
	\togglefalse{book}
}
\newcommand{\setbook}{
	\togglefalse{instructor}
	\togglefalse{standard}
	\togglefalse{solutions}
	\toggletrue{book}
	\setbookgeometry
}

%
% Infer the document level from the \jobname
%
\usepackage{xstring}
\IfSubStr{\jobname}{\detokenize{book}}{\setbook}{
	\IfSubStr{\jobname}{\detokenize{solutions}}{\setsolutions}{
		\IfSubStr{\jobname}{\detokenize{instructor}}{\setinstructor}{
			\setstandard
		}
	}
}

%
% Hide the non-problem environments
%
\newcommand{\coversubtitle}{} % we override the subtitle in each mode, so make sure the command exists to override.
\iftoggle{instructor}{
	\voidenvironment{module}
	\voidenvironment{bookonly}
	\voidenvironment{displayonly}
	\renewcommand{\coversubtitle}{Instructor Guide}
}{}
\iftoggle{solutions}{
	\voidenvironment{module}
	\voidenvironment{bookonly}
	\voidenvironment{displayonly}
	\voidenvironment{lesson}
	\voidenvironment{notes}
	\renewcommand{\coversubtitle}{Solutions}
}{}
\iftoggle{standard}{
	\voidenvironment{module}
	\voidenvironment{bookonly}
	\voidenvironment{solution}
	\voidenvironment{annotation}
	\voidenvironment{lesson}
	\renewcommand{\coversubtitle}{MAT335 Notes}
}{}
\iftoggle{book}{
	\voidenvironment{displayonly}
	\voidenvironment{solution}
	\voidenvironment{annotation}
	\voidenvironment{lesson}
	\renewcommand{\coversubtitle}{{\hspace{-5pt}\begin{tabular}{l}MAT223 Workbook\\\small\today{} Edition\end{tabular}}}
}{}
%\voidenvironment{solution}
%\voidenvironment{annotation}
%\voidenvironment{lesson}
%%\voidenvironment{notes}
%%\voidenvironment{displayonly}




\begin{document}

%%
%% End Definitions
%%






\pagestyle{empty}

\begin{tikzpicture}[remember picture,overlay, shift={(current page.north west)}, >=latex]
	\definecolor{coverblue}{HTML}{ffd33c}
	\definecolor{coverpink}{HTML}{ff97e8}
	\definecolor{coveraccentpink}{HTML}{ffd33c}
	\definecolor{coverorange}{HTML}{ffffff}
	\definecolor{covershade}{HTML}{3f004d}



	\begin{scope}
	%	\node[anchor=south east,inner sep=0pt,outer sep=0pt,] 
	%	at (current page.south east) {\includegraphics[width=\paperwidth, height=\paperheight]{images/orangeback.jpg}};

		\fill[path fading=north,covershade] (0,2.2in) rectangle ([yshift=-1.01in, xshift=2pt]current page.north east);
		\fill[covershade] (0,-1in) rectangle ([yshift=-2.05in]current page.north east);
		\fill[path fading=south, covershade] (0,-2in) rectangle ([xshift=1in,yshift=2in]current page.south east);
	\end{scope}


  \begin{scope}[yscale=-1, xscale=1, x=2.7pt, y=2.7pt,line join=miter,line cap=butt,line width=1.3pt, yshift=2.3cm, xshift=.7cm,
	  ]

	  \coordinate (E) at (57.55,23.15);
	  \coordinate (A) at (195.3,27.9);
	  \coordinate (C) at (195.3,104);
	  \coordinate (SUB) at (141, 30);
		
  \end{scope}

  	\coordinate (X) at (3, -15);
	\path (E) --  (C)  
		coordinate[pos=0] (X1)
		coordinate[pos=.017] (X2)
		coordinate[pos=.08] (X3)
		coordinate[pos=.2] (X4)
		coordinate[pos=.5] (X5)
		coordinate[pos=1] (X6)
		-- (A);
	\draw[coverpink, line width=1.3pt] (E) --  (C)  
		-- (A);
	\begin{scope}[coverorange, line width=0.5pt]
		\draw[->] 
			(X) -- (X1) node[pos=.5, above left] {\Large $\vec u$};
		\draw[->] 
			(X) -- (X2);
		\draw[->] 
			(X) -- (X3);
		\draw[->] 
			(X) -- (X4);
		\draw[->, coveraccentpink, line width=1.3pt] 
			(X) -- (X5) node[pos=.6, below right] {\Large $\vec w_{\alpha}=\alpha {\color{coverorange}\vec u} + (1-\alpha){\color{coverorange}\vec v}$};
		\draw[->] 
			(X) -- (X6) node[pos=.7, below right] {\Large $\vec v$};
	\end{scope}

	\path[white] (SUB) node[anchor=north west] {\Large \bfseries \sffamily \coversubtitle};

\newcommand{\authornames}{\huge \sffamily \bfseries \begin{tabular}{r}Jason Siefken\end{tabular}}
	\newcommand{\ypadd}{.5em}
	\newcommand{\xpadd}{1em}

	\draw (0, -24) node[right, xshift=10em] (AUTHOR) {\phantom{\authornames}};
	\path let \p1 = (AUTHOR.north) in coordinate (Ab1) at (0,\y1+\ypadd);
	\path let \p1 = (AUTHOR.north east) in coordinate (Ab2) at (\x1+\xpadd,\y1+\ypadd);
	\path let \p1 = (AUTHOR.south east) in coordinate (Ab3) at (\x1+\xpadd,\y1-\ypadd);
	\path let \p1 = (AUTHOR.south) in coordinate (Ab4) at (0,\y1-\ypadd);

	\path[fill=covershade, path fading=west, opacity=.8] (Ab1) -- (Ab2) -- (Ab3) -- (Ab4);
	\draw[covershade!80!black, line width=1.3pt] (Ab1) -- (Ab2) -- (Ab3) -- (Ab4);
	\draw (0, -24) node[right, xshift=10em, white] (AUTHOR) {\authornames};

\end{tikzpicture}


\newpage


\newcommand{\standardintro}{
\begin{center}
{\huge\bf Inquiry Based Linear Algebra}\\

\vspace{.7in}
{
\it \copyright\,Jason Siefken, 2016--2019 \\
Creative Commons By-Attribution Share-Alike\, \makebox(30,5){\includegraphics[height=1.2em]{by-sa.pdf}}
}
\end{center}

\section*{About the Document}

This document is a hybrid of many linear algebra resources, including those of
the IOLA (Inquiry Oriented Linear Algebra) project and Jason Siefken's
IBL Linear Algebra project.

This document is a mix of short problems and more involved exploratory question. A typical
class day looks like:
\begin{enumerate}
	\item {\bf Introduction by instructor.} This may involve giving a definition,
		a broader context for the day's topics, or answering questions.

	\item {\bf Students work on problems.} Students work individually or in pairs/small groups
		on the prescribed problem. During this time the instructor moves
		around the room addressing questions that students may have and giving
		one-on-one coaching.

	\item {\bf Instructor intervention.} When most students have successfully solved
		the problem, the instructor refocuses the class by providing an
		explanation or soliciting explanations from students.
		This is also time for the instructor to ensure that everyone has
		understood the main point of the exercise (since it is sometimes
		easy to miss the point!).

		If students are having trouble, the instructor can give hints
		and additional guidance to ensure students' struggle is productive.

	\item {\bf Repeat step 2.}
\end{enumerate}

Using this format, students are thinking (and happily so) most of the class. Further,
after struggling with a question, students are especially primed to hear the insights of the instructor.

These problems are geared towards concepts instead of computation, though some problems
focus on simple computation. The questions also have a geometric lean. Vectors are initially
introduced with familiar coordinate notation, but eventually, coordinates are understood to be
\emph{representations} of vectors rather than ``true'' geometric vectors, and objects like the
determinant are defined via oriented volumes rather than formulas involving matrix entries.

\bigskip
{\bf License} Unless otherwise mentioned, pages of this document are licensed under
the Creative Commons By-Attribution Share-Alike License. That means, you are free
to use, copy, and modify this document provided that you provide attribution to the
previous copyright holders and you release your derivative work under the same license.
Full text of the license is at \url{http://creativecommons.org/licenses/by-sa/4.0/}

If you modify this document, you may add your name to the copyright list. Also,
if you think your contributions would be helpful to others, consider making a
pull request, or opening an \emph{issue} at \url{https://github.com/siefkenj/IBLLinearAlgebra}

Content from other sources is reproduced here with permission and retains the
Author's copyright. Please see the footnote of each page to verify the
copyright.

\newpage
}


\setcounter{page}{1}
\pagestyle{siefken}


\addcontentsline{toc}{chapter}{Lessons}

	\section*{Dynamical Systems}

	\begin{definition}[Dynamical System]
		A pair $(T,X)$ is called a \emph{dynamical system} if
		$X$ is a set and $T:X\to X$ is a function. In the context of
		a dynamical system, $T$ is often called a \emph{transformation}.
	\end{definition}

	This definition is very general, and most things you encounter could be considered
	a dynamical system. For example, if $X=\{\text{air molecuels and their positions on earth}\}$
	and $T:X\to X$ to be the result of the wind blowing for one second, then $(T,X)$ is a dynamical system.
	Alternatively, we could take the state of your computer's memory (RAM) to be a set and your processor executing a single
	instruction to be a transformation.

	It's hard to say much about general dynamical systems.
	However, throughout the course, we will find ways to classify dynamical systems. Once we ``narrow the field'',
	we'll be able to say lot's of interesting things.

	\subsection*{Newton's Method}
	Newton's method is a way of using tangent-line approximations to functions to estimate their roots.
	It is an iterative procedure\footnote{ Whenever something is iterative, you should think dynamics!}.

	\question
	Let $f:\R\to\R$ be a differentiable function and let $T_f:\{\text{guesses}\}\to\{\text{guesses}\}$ be
	a single application of Newton's method.

	\begin{parts}
		\item Find a general formula for $T_f$.
		\item Let $f(x)=x(x-2)(x-3)$. Compute $T_f^n(4)$ for $n=0,1,2,3$.
		\item Do you think
			\[
				\lim_{n\to\infty} T_f^n(4)
			\]
			converges? If so, what does it converge to? Can you prove your answer?
	\end{parts}

	\begin{definition}[Fixed Point]
		Let $(T,X)$ be a dynamical system. A point $a\in X$ is called a \emph{fixed point} if
		$T(a)=a$.
	\end{definition}

	\begin{definition}[Basin of Attraction]
		Let $(T,X)$ be a dynamical system and let $x\in X$. The \emph{basin of attraction}
		of $x$ is the set
		\[
			A_x=\Set{y\in X: \lim_{n\to\infty} T^ny=x}.
		\]
	\end{definition}
	Eventually, we will talk about more general \emph{basins of attraction}, but for now we will limit ourselves to
	that of a single point.

	\question
	Let $f(x)=x(x-2)(x-3)$ and let $T_f$ be the function that applies a single iteration of Newton's method (as before).
	\begin{parts}
		\item fsd
	\end{parts}

\end{document}
